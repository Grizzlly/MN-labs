\documentclass{exam}

\usepackage{indentfirst}
\usepackage{graphicx}
\usepackage{listings}
\usepackage{color}
\usepackage{fancyvrb}
\usepackage{url}
\usepackage{amsmath}

\definecolor{mygreen}{rgb}{0,0.6,0}
\definecolor{mygray}{rgb}{0.5,0.5,0.5}
\definecolor{mymauve}{rgb}{0.58,0,0.82}

\lstset{ %
	backgroundcolor=\color{white},		% choose the background color; you must add \usepackage{color} or \usepackage{xcolor}
	basicstyle=\small\ttfamily,		% the size of the fonts that are used for the code
	breakatwhitespace=false,			% sets if automatic breaks should only happen at whitespace
	breaklines=true,					% sets automatic line breaking
	captionpos=b,						% sets the caption-position to bottom
	columns=fullflexible,
	commentstyle=\color{mygreen},		% comment style
	deletekeywords={...},				% if you want to delete keywords from the given language
	escapeinside={\%*}{*)},			% if you want to add LaTeX within your code
	extendedchars=true,				% lets you use non-ASCII characters; for 8-bits encodings only, does not work with UTF-8
	frame=single,						% adds a frame around the code
	keepspaces=true,					% keeps spaces in text, useful for keeping indentation of code (possibly needs columns=flexible)
	keywordstyle=\color{blue},			% keyword style
	language=Octave,					% the language of the code
	morekeywords={*,...},				% if you want to add more keywords to the set
	%   numbers=left,						% where to put the line-numbers; possible values are (none, left, right)
	%   numbersep=6pt,						% how far the line-numbers are from the code
	%   numberstyle=\tiny\color{mygray},	% the style that is used for the line-numbers
	rulecolor=\color{black},			% if not set, the frame-color may be changed on line-breaks within not-black text (e.g. comments (green here))
	showspaces=false,					% show spaces everywhere adding particular underscores; it overrides 'showstringspaces'
	showstringspaces=false,			% underline spaces within strings only
	showtabs=false,					% show tabs within strings adding particular underscores
	stepnumber=1,						% the step between two line-numbers. If it's 1, each line will be numbered
	stringstyle=\color{mymauve},		% string literal style
	tabsize=2,							% sets default tabsize to 2 spaces
	title=\lstname						% show the filename of files included with \lstinputlisting; also try caption instead of title
}

% Creates a new command to include a perl script, the first parameter is the filename of the script (without .pl), the second parameter is the caption
\newcommand{\octavescript}[2]{
	\lstinputlisting[caption=#2,label=#1]{#1}
}

\newcommand{\MNLab}{Laborator\ \#5}
\newcommand{\MNLabTitle}{Metode iterative pentru rezolvarea sistemelor de ecuații liniare: Jacobi, Gauss-Siedel, Suprarelaxare}
\newcommand{\MNLabTitleHeader}{Metode iterative}
\newcommand{\MNAuthor}{Andrei STAN, Bogdan Țigănoaia}

\renewcommand{\contentsname}{Cuprins}
\renewcommand{\figurename}{Figura}

\setlength{\parskip}{0.5\baselineskip}

\graphicspath{{./img/}}

\title{
	\textmd{\textbf{\MNLabTitle}}
	\author{Colaboratori: \MNAuthor}
}

\pagestyle{headandfoot}

\header{Metode Numerice}
{\MNLabTitleHeader, Pagina \thepage\ din \numpages}
{2025}
\footer{Facultatea de Automatică și Calculatoare}{}{Pagina \thepage\ din \numpages}

\begin{document}

\begin{coverpages}

	\maketitle
	\tableofcontents

\end{coverpages}

\section{Obiective laborator}

\par \^{I}n urma parcurgerii acestui laborator, studentul va fi capabil s\u{a} rezolve sisteme de ecua\c{t}ii liniare utiliz\^{a}nd metode iterative.

\section{Noțiuni teoretice}

\par Metodele exacte de rezolvare a sistemelor de ecua\c{t}ii liniare, av\^{a}nd complexitate O(${n}^{3}$), au aplicabilitate limitat\u{a} la ordine de sisteme ce nu dep\u{a}\c{s}esc 1000. Pentru sisteme de dimensiuni mai mari se utilizeaz\u{a} metode cu complexitate O(${n}^{2}$) \^{i}ntr-un singur pas de itera\c{t}ie. Acestea utilizeaz\u{a} rela\c{t}ii de recuren\c{t}\u{a}, care prin aplicare repetat\u{a} furnizeaz\u{a} aproxima\c{t}ii, cu precizie controlat\u{a}, a solu\c{t}iei sistemului.


Metodele iterative transform\u{a}  sistemul $Ax = b$  \^{i}n $x = Gx + c$. Pornindu-se cu o aproxima\c{t}ie ini\c{t}ial\u{a} ${x}^{(0)}$ a solu\c{t}iei, rela\c{t}ia de recuren\c{t}\u{a}  folosit\u{a} are forma:
$${x}^{(p+1)} = G{x}^{(p)} + c$$
unde:
\begin{itemize}
	\item ${x}^{(0)}, {x}^{(1)}, ... , {x}^{(p)}, ...$ sunt aproxim\u{a}rile solu\c{t}iei;
	\item $G$ reprezint\u{a} matricea de itera\c{t}ie;
	\item $c$ reprezint\u{a} vectorul de itera\c{t}ie.
\end{itemize}

\par O metod\u{a} este convergent\u{a} dac\u{a} este stabil\u{a} \c{s}i consistent\u{a}. Condi\c{t}ia necesar\u{a} \c{s}i suficient\u{a} de convergen\c{t}\u{a} este:
$${\rho}(G) < 1$$
unde ${\rho}(G) = \max(|{\lambda}_{1}|, |{\lambda}_{2}|, ... , |{\lambda}_{n}|)$ reprezint\u{a} raza spectral\u{a} a matricei de itera\c{t}ie $G$ \c{s}i ${\lambda}_{i}, i = 1:n$ reprezint\u{a} valorile proprii ale matricei.

\par Metodele iterative se bazeaz\u{a} pe descompunerea matricei $A$ sub forma $A = N - P$. Atunci sistemul devine:

$(N - P)x = b, $ adică $x = {N}^{-1}Px + {N}^{-1}b.$

Astfel, rezult\u{a} rela\c{t}ia de recuren\c{t}\u{a}:
$${x}^{(p+1)} = {N}^{-1}P{x}^{(p)} + {N}^{-1}b$$
de unde putem identifica $G = {N}^{-1}P$ \c{s}i $c = {N}^{-1}b$.

\par Se parti\c{t}ioneaz\u{a} matricea $A$ pun\^{a}nd \^{i}n eviden\c{t}\u{a} o matrice diagonal\u{a} $D$, o matrice strict triunghiular inferioar\u{a} $L$ \c{s}i o matrice strict triunghiular superioar\u{a} $U$:
$$A = D - L - U.$$

\subsection{Metoda Jacobi}
\par \^{I}n metoda Jacobi se aleg:

$N = D$

$P = L + U$

${G}_{J} = {D}^{-1}(L + U)$

\par Solu\c{t}ia sistemului este:
$${x}_{i}^{(p+1)} = \frac{{b}_{i} - \sum_{j = 1, j \neq i}^{n}{a}_{ij}{x}_{j}^{(p)}}{{a}_{ii}}$$

%\par Metoda converge dac\u{a} matricea $A$ este diagonal dominant\u{a}\footnote{\url{http://en.wikipedia.org/wiki/Diagonally\_dominant\_matrix}}  pe linii ($|{a}_{ii}| > \sum_{j \neq i}|{a}_{ij}| )$.

\subsection{Metoda Gauss-Seidel}
\par La aceast\u{a} metod\u{a} se aleg:

$N = D - L$

$P = U$

${G}_{GS} = {(D - L)}^{-1}U$

Solu\c{t}ia sistemului este:
$${x}_{i}^{(p+1)} = \frac{{b}_{i} - \sum_{j = 1}^{i-1}{a}_{ij}{x}_{j}^{(p+1)} - \sum_{j = i + 1}^{n}{a}_{ij}{x}_{j}^{(p)}}{{a}_{ii}}$$

%\par Metoda converge dac\u{a} matricea A este simetric\u{a} (${a}_{ij} = {a}_{ji},  i,j = 1:n$) \c{s}i pozitiv definit\u{a} ($\forall z \in R^{n}, {z}^{T}Az >0$).

Observa\c{t}ii:
\begin{enumerate}
	\item  Dac\u{a} matricea sistemului este diagonal dominant\u{a} pe linii, metoda Gauss Seidel este convergent\u{a}. Reciproca nu este adevarat\u{a}.
	\item  O matrice $A$ este diagonal dominant\u{a} pe linii dac\u{a} \c{s}i numai dac\u{a} are urm\u{a}toarea proprietate: pentru fiecare linie i, modulul elementului de pe diagonala principal\u{a}, $A(i,i)$ este strict mai mare dec\^{a}t suma modulelor elementelor de pe aceea\c{s}i linie $i$.
\end{enumerate}

\subsection{Metoda suprarelax\u{a}rii}
Pentru g\u{a}sirea unei descompuneri c\^{a}t mai rapid convergente, se introduce parametrul de relaxare $\omega$:
$$A = N - P = N - \omega N - P + \omega N = (1 - \omega)N - (P - \omega N) = N(\omega) - P(\omega)$$

de unde ob\c{t}inem:

$N(\omega) = (1 - \omega)N$

$P(\omega) = P - \omega N$

$G(\omega) = {N}^{-1}(\omega)P(\omega) = \frac{{N}^{-1}}{1-\omega}(P - \omega N) = \frac{{N}^{-1}P - \omega{I}_{n}}{1-\omega}$

Condi\c{t}ia de stabilitate impune $\omega \in (0,2)$. \^{I}n practic\u{a} se face o alt\u{a} alegere, astfel:

$N(\omega) = \frac{1}{\omega}D - L, \quad P(\omega) = (\frac{1}{\omega} - 1)D + U, \quad {G}_{\omega} = (D-\omega L)^{-1}[(1-\omega)D+\omega U]$

Solu\c{t}ia sistemului se poate scrie sub forma:
$$x_{i}^{(p+1)}=\omega\frac{{b}_{i} - \sum_{j = 1}^{i-1}{a}_{ij}{x}_{j}^{(p+1)} - \sum_{j = i + 1}^{n}{a}_{ij}{x}_{j}^{(p)}}{{a}_{ii}}+(1-\omega)x_{i}^{(p)}$$

Dac\u{a} se alege $\omega = 1 \Rightarrow$ metoda Gauss-Seidel.


\section{Probleme rezolvate}

\subsection{Problema 1}

S\u{a} se rezolve sistemul folosind metoda Gauss-Seidel:
$$	\left\{
	\begin{array}{ccccccc}
		7x_1 & + & 2x_2 & - & 4x_3 & = & 7  \\
		3x_1 & + & 6x_2 & + & 2x_3 & = & 15 \\
		2x_1 & - & 5x_2 & + & 8x_3 & = & 28 \\
	\end{array} \right.
$$

\textit{Solu\c{t}ie}:

Scriem formulele de recuren\c{t}\u{a}

$$\left\{
	\begin{array}{ccccccccc}
		x_1^{(k+1)} & = & -2/7x_2^{(k)} & + & 4/7x_3^{(k)} & + & 7/7  \\
		x_2^{(k+1)} & = & -3/6x_1^{(k)} & - & 2/6x_3^{(k)} & + & 15/6 \\
		x_3^{(k+1)} & = & -2/8x_1^{(k)} & + & 5/8x_2^{(k)} & + & 28/8 \\
	\end{array} \right.
$$

Dac\u{a} alegem $x_{1}^{(0)} = x_{2}^{(0)}=x_{3}^{(0)}=0$ ob\c{t}inem urm\u{a}toarele rezultate:
$$  \begin{array}{c||ccc}
		k & x_1^{(k)} & x_2^{(k)} & x_3^{(k)} \\
		\hline
		0 & 0         & 0         & 0         \\
		1 & 1         & 2         & 4.5       \\
		2 & 3.00      & -0.5      & 2.43      \\
		3 & 2.53      & 0.41      & 3.12      \\
		4 & 2.66      & 0.12      & 2.91      \\
		5 & 2.62      & 0.21      & 2.97      \\
	\end{array}
$$

Solu\c{t}ia exact\u{a} este: $x_{1}=2.63,\ x_{2}=0.19\ x_{3}=2.96$.
%\end{Problem}

\subsection{Problema 2}
Folosi\c{t}i metoda Jacobi pentru a aproxima solu\c{t}ia sistemului:
$$	\left\{
	\begin{array}{ccccccc}
		10x_1 & - & 5x_2  & + & x_3   & = & 1 \\
		x_1   & + & 4x_2  & + & 3 x_3 & = & 4 \\
		4x_1  & - & 3 x_2 & - & 9x_3  & = & 6 \\
	\end{array} \right.
$$

\textit{Solu\c{t}ie}:

Scriem formulele de recuren\c{t}\u{a}
$$\left\{
	\begin{array}{ccccccc}
		x_1^{(k+1)} & = & 5/10x_2^{(k)} & - & 1/10x_3^{(k)} & + & 1/10 \\
		x_2^{(k+1)} & = & -1/4x_1^{(k)} & - & 3/4x_3^{(k)}  & + & 4/4  \\
		x_3^{(k+1)} & = & 4/9x_1^{(k)}  & - & 3/9x_2^{(k)}  & - & 6/9  \\
	\end{array} \right.
$$

Aleg\^{a}nd $x_{1}^{(0)}=x_{2}^{(0)}=x_3^{(0)}=0\Rightarrow$


$$  \begin{array}{c||ccc}
		k & x_1^{(k)} & x_2^{(k)} & x_3^{(k)} \\
		\hline
		0 & 0         & 0         & 0         \\
		1 & 0.1       & 1.00      & -0.66     \\
		2 & 0.66      & 1.47      & -1.95     \\
		3 & 0.93      & 1.55      & -0.86     \\
		4 & 0.96      & 1.41      & -0.76     \\
		5 & 0.88      & 1.33      & -0.71     \\
	\end{array}
$$

Solu\c{t}ia exact\u{a} este: $x_{1}=0.84,\ x_{2}=1.34,\ x_{3}=-0.73$.
%\end{Problem}

\subsection{Problema 3}
Fie sistemul $Ax=b, A \in R^{2\times 2}, x, b \in R^{2},
	A =
	\left[ {\begin{array}{cc}
					2 & 2 \\
					1 & 3
				\end{array} } \right]
$. Matricea $A$ nu este diagonal dominant\u{a} pe linii. \^{I}n aceste condi\c{t}ii este convergent\u{a} metoda Gauss-Seidel?

\textit{Solu\c{t}ie}:

Se determin\u{a} matricea de itera\c{t}ie a sistemului pentru metoda Gauss-Seidel, ${G}_{GS}$.

$A = D - L - U \Rightarrow \quad D = \left[ {\begin{array}{cc}
					2 & 0 \\
					0 & 3
				\end{array} } \right], \quad L =
	\left[ {\begin{array}{cc}
					0  & 0 \\
					-1 & 0
				\end{array} } \right], \quad U =
	\left[ {\begin{array}{cc}
					0 & -2 \\
					0 & 0
				\end{array} } \right]$

Atunci:

${G}_{GS} = {(D-L)}^{-1}U =
	\left[ {\begin{array}{cc}
					2 & 0 \\
					1 & 3
				\end{array} } \right] ^ {-1}
	\left[ {\begin{array}{cc}
					0 & -2 \\
					0 & 0
				\end{array} } \right]
	=
	\left[ {\begin{array}{cc}
					0 & -1          \\
					0 & \frac{1}{3}
				\end{array} } \right]$.

$\det{(\lambda I - G_{GS})} = \left| {\begin{array}{cc}
		\lambda & 1                     \\
		0       & \lambda - \frac{1}{3}
	\end{array} } \right| = 0\Rightarrow \lambda(G_{GS}) = \{0,\frac{1}{3}\}$  \c{s}i $\rho(G_{GS}) = \frac{1}{3} < 1$.

$\Rightarrow$ metoda Gauss-Seidel este convergent\u{a}.
%\end{Problem}


\subsection{Problema 4}
S\u{a} se implementeze o func\c{t}ie OCTAVE care rezolv\u{a} un sistem de ecua\c{t}ii liniare folosind metoda iterativ\u{a} Gauss-Seidel. Date de intrare: $A$ - matricea sistemului; $b$ - vectorul termenilor liberi; $x0$ - aproxima\c{t}ia ini\c{t}ial\u{a} a solu\c{t}iei; $tol$ - precizia determin\u{a}rii solu\c{t}iei; $maxiter$ - num\u{a}rul maxim de itera\c{t}ii. Date de ie\c{s}ire: $x$ - solu\c{t}ia sistemului; $succes$ - variabilă care indică convergenţa metodei.

\textit{Solu\c{t}ie}:

\octavescript{./src/GaussSeidel.m}{}

\begin{center}
	\begin{tabular}{| l | l |}
		\hline
		\textbf{Date de intrare:}  & \textbf{Date de ieşire:} \\
		$A = $
		$\left[ {\begin{array}{cc}
							         2 & 2 \\
							         1 & 3
						         \end{array} } \right],$

		$b = $
		$\left[ {\begin{array}{cc}
							         2 \\
							         1
						         \end{array} } \right],$

		$x0 =
			\left[ {\begin{array}{cc}
							        0 \\
							        0
						        \end{array} } \right],$

		$ tol=0.0001, maxiter=100$ &
		$x = $
		$\left[ {\begin{array}{cc}
							         1 \\
							         0
						         \end{array} } \right]$
		\\
		\hline
	\end{tabular}
\end{center}
%\end{Problem}

\section{Probleme propuse}

%--------------------------------------------------------------------------------------
%	PROBLEM 1
%--------------------------------------------------------------------------------------
\subsection{Problema 1}
Fie sistemul $Ax=b, A \in R^{2 \times 2}, b \in R^{2}$, cu
$A =\left[ {\begin{array}{cc}
					2  & -1 \\
					-1 & 2
				\end{array} } \right]$. Determina\c{t}i raza spectral\u{a} a matricei de itera\c{t}ie Jacobi. Stabili\c{t}i convergen\c{t}a metodei Jacobi.
%\end{Problem}

%----------------------------------------------------------------------------------------
%       PROBLEM 2
%----------------------------------------------------------------------------------------

% To have just one problem per page, simply put a \clearpage after each problem

\subsection{Problema 2}
Fie sistemul liniar:
$$	\left\{
	\begin{array}{ccccccc}
		2x & + & y  & + & z  & = & 4 \\
		x  & + & 2y & + & z  & = & 4 \\
		x  & + & y  & + & 2z & = & 4 \\
	\end{array} \right.
$$

Stabili\c{t}i:
%\begin{enumerate}

a) dac\u{a} matricea sistemului este diagonal dominant\u{a} pe linii;

b) convergen\c{t}a metodei Jacobi;

c) convergen\c{t}a metodei Gauss-Seidel.
%\end{enumerate}

\^{I}n caz de convergen\c{t}\u{a}, calcula\c{t}i solu\c{t}ia iterativ\u{a} dup\u{a} trei pa\c{s}i. Alege\c{t}i voi aproxima\c{t}ia ini\c{t}ial\u{a}.

%\end{Problem}

%----------------------------------------------------------------------------------------
%       PROBLEM 3
%----------------------------------------------------------------------------------------
\subsection{Problema 3}
Fie o matrice $A \in R^{n\times n}$ tridiagonal\u{a} \footnote{\url{http://mathworld.wolfram.com/TridiagonalMatrix.html}} \c{s}i sistemul de ecua\c{t}ii $Ax=b$, cu $b,\ x \in R^{n}$. Scrie\c{t}i o func\c{t}ie OCTAVE care rezolv\u{a} sistemul de ecua\c{t}ii prin metoda iterativ\u{a} Jacobi.

\octavescript{./src/Jacobi.m}{Algoritmul Jacobi}
%\end{Problem}

%-----------------------------------------------------------------------------------------
%       PROBLEM 4
%-----------------------------------------------------------------------------------------
\subsection{Problema 4}
S\u{a} se implementeze o func\c{t}ie OCTAVE care rezolv\u{a} un sistem liniar de ecua\c{t}ii folosind metoda suprarelax\u{a}rii.
\octavescript{./src/sor.m}{Suprarelaxare}

S\u{a} se testeze func\c{t}ia folosind diferite valori pentru $\omega$.
%\end{Problem}       
\end{document}
